% question one
\question{1}{0.7 points}

\note{A central question in labor economics is: \textbf{How much more do individuals earn with higher levels of education?} Economists often estimate the \emph{returns to education}---that is, the increase in earnings associated with completing high school, college, or additional years of schooling.}

\note{Using the panel data provided, begin by constructing a \textbf{bar chart} showing \textbf{mean income by education group}. Group individuals based on their \textbf{highest level of educational attainment} (e.g., less than high school, high school graduate, some college, college degree or more), and plot the \textbf{average income} for each category.}

\fig{fig11.png}{Mean income by education level}

\note{Recent debates around student debt and the value of higher education often assume that education “pays off” equally for everyone. \textbf{Does your analysis support that assumption?} To explore this, create \textbf{separate plots by gender} to highlight any differences in the relationship between education and earnings. Discuss your findings.}

\note{\textbf{Note:} For this question, create and use a categorical education variable based on each individual's highest level of education completed across the panel. Construct four categories:
\begin{itemize}
    \vspace{-1em}
	\item Less than high school (11 or fewer years)
	\vspace{-1em}
	\item High school graduate (exactly 12 years)
	\vspace{-1em}
    \item Some college (13 to 15 years)
	\vspace{-1em}
    \item College degree or more (16 or more years)
\end{itemize}}

\fig{fig12.png}{Mean income by education level, by gender}