% question
\question{8}{0.75 points}

\note{Finally, revisit your data and evaluate whether \textbf{attrition} is present in your sample. Based on your preferred model, discuss the likelihood of \textbf{attrition bias}. What conclusions can you draw regarding its presence, and how might it affect the validity of your results?}

\vspace{-1em}

\begin{minted}{stata}
xtset pid wave
xtdescribe
bysort pid (wave): gen n_waves = _N
gen all_waves = n_waves == 17
xtreg log_income i.child_birth##i.male age ib0.mstatus ib4.ethnicity ///
    edyears age_mean mstatus_mean all_waves, re
\end{minted}

\vspace{-1em}

\tab{tab81.tex}{Attrition bias: \emph{all waves} indicator}

\vspace{-1em}

The \emph{all waves} indicator has a positive coefficient of 0.0591, statistically significant at the 5\% significance level. This indicates that individuals who stayed for all waves had a 6.09\% higher income than those who dropped out at some point, ceteris paribus.

\vspace{-1em}

\begin{minted}{stata}
bysort pid (wave): gen next_wave = (wave[_n+1] == wave + 1)
xtreg log_income i.child_birth##i.male age ib0.mstatus ib4.ethnicity ///
    edyears age_mean mstatus_mean next_wave, re
\end{minted}

\vspace{-1em}

\tab{tab82.tex}{Attrition bias: \emph{next wave} indicator}

The \emph{next wave} indicator has a positive coefficient of 0.0591, statistically significant at the 1\% significance level. This indicates that individuals who appear in the subsequent wave have 6.87\% higher income in the current period compared to those who drop out before the next wave, ceteris paribus.

\begin{minted}{stata}
xtreg log_income i.child_birth##i.male age ib0.mstatus ib4.ethnicity ///
    edyears age_mean mstatus_mean n_waves, re
\end{minted}

The \emph{number of waves} indicator has a positive coefficient of 0.0070, statistically significant at the 5\% significance level. This indicates that for each additional wave an individual has been part of, they will have 0.7\% higher income than the baseline, ceteris paribus.

All three attrition indicators are statistically significant, providing strong evidence that attrition is present in the sample. The sign and magnitude across the indicators show that lower-income individuals are more likely to drop out of the panel, meaning the sample becomes progressively less representative of lower-income populations over time. If women who experience larger child penalties are more likely to drop out, we may be underestimating the true child penalty; in other words, attrition would introduce downward bias in our model.

\tab{tab83.tex}{Attrition bias: \emph{number of waves} indicator}