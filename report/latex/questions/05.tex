% question
\question{5}{0.9 points}

\note{Next, estimate a \textbf{Correlated Random Effects (CRE)} model to examine the effect of years of education (edyears) on \textbf{log(income)}.}

\begin{minted}{stata}
by pid: egen edyears_mean = mean(edyears)
by pid: egen age_mean = mean(age)
by pid: egen male_mean = mean(male)
by pid: egen mstatus_mean = mean(mstatus)
by pid: egen ethnicity_mean = mean(ethnicity)
by pid: egen child_birth_mean = mean(child_birth)

xtreg log_income edyears age i.male ib0.mstatus ib4.ethnicity ///
    i.child_birth age_mean mstatus_mean child_birth_mean ///
    edyears_mean male_mean ethnicity_mean, re
\end{minted}

According to the CRE model, for each additional year of schooling individuals receive a 17.43\% higher average income, ceteris paribus. This difference is statistically significant at the 1\% significance level.

\subquestion{a}{What is one advantage of the \textbf{CRE estimator} compared to the \textbf{random effects (RE)} estimator?}

While RE requires that individual effects be uncorrelated with the explanatory variables, CRE models the correlation between individual effects and time-varying explanatory variables, making it more robust to violations of the random effects assumption.

\tab{tab51.tex}{Correlated Random Effects (CRE) model}

\subquestion{b}{What is one advantage of the \textbf{CRE estimator} compared to the \textbf{fixed effects (FE)} estimator?}

While FE models remove all time-invariant characteristics through the within transformation, CRE retains the ability to estimate effects of variables that don't change over time, such as gender, race, or place of birth. By modelling the correlation structure between individual effects and explanatory variables, CRE becomes useful when estimating the effects of time-constant characteristics.

\subquestion{c}{Compare the estimated coefficient for years of education from the \textbf{CRE model} with those from the \textbf{RE} and \textbf{FE} models. Are the coefficients similar or different? Explain why this is the case.}

Compared to the previous models, CRE has a coefficient similar to the FE estimate (0.1607 compared to 0.1610) as well as the similar standard error (0.0039 compared to 0.0038). This suggests that CRE succesfully replicates the FE estimate's ability to handle endogeneity without having to omit time-invariant variables.

Compared to the RE estimate, the CRE coefficient is notably higher than the RE coefficient (0.1607 compared to 0.1509), which is similar to the difference between RE and FE. The higher FE coefficient indicates that there is indeed correlation between individual effects and education, confirming the Hausman test results. While CRE and FE take this correlation into account, the RE estimate's lower coefficient reflects downward bias from failing to account for this correlation.

All in all, the CRE estimate is very similar to the FE estimate, addressing the correlation between individual effects and regressors detected by the Hausman test. The key advantage of CRE over FE is that it can estimate time-invariant characteristics, making it appropriate for handling endogeneity when time-invariant variables cannot be omitted.

\subquestion{d}{Based on your CRE estimates, does the assumption of \textbf{exogeneity} appear to hold? Which estimator would you consider most appropriate in this context?}

According to the model, the \icode{mstatus_mean} CRE estimator is statistically significant at the 1\% significance level. This suggests that the exogeneity assumption does not hold.

To most accurately estimate \emph{returns on education}, the FE model would make most sense as it is unambiguously robust; however, if time-invariant effects had to be estimated, CRE would provide unbiased estimates while allowing for time-invariant variables.

\tab{comp5.tex}{POLS, RE, FE, CRE comparison}