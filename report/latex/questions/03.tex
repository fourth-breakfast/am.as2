% question
\question{3}{0.5 points}

\note{So far, the panel structure of the data has been largely unexploited. Random effects (RE) estimation can improve the efficiency of the estimates compared to pooled OLS.}

\subquestion{a}{Estimate the effect of years of education (edyears) on \textbf{log(income)} using the \textbf{random effects} (RE) model, controlling for age, gender (male), marital status categories, ethnicity categories, and childbirth. Interpret the estimated coefficient for years of education in terms of its \textbf{sign}, \textbf{magnitude}, and \textbf{statistical significance}. Then, compare the RE estimate and standard error of the education coefficient with those obtained from the \textbf{pooled OLS model}.}

\begin{minted}{stata}
xtset pid wave
\end{minted}

\begin{minted}{stata}
xtreg log_income edyears age i.male ib0.mstatus ib4.ethnicity ///
    i.child_birth, re
estimates store random
\end{minted}

\tab{tab31.tex}{Random effects (RE) model}

\tab{comp3.tex}{POLS, RE comparison}

\subquestion{b}{Under which conditions and why can the random effects estimator be \textbf{more efficient} than pooled OLS?}