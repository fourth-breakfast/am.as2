% question
\question{5}{0.9 points}

\note{Next, estimate a \textbf{Correlated Random Effects (CRE)} model to examine the effect of years of education (edyears) on \textbf{log(income)}.}

\begin{minted}{stata}
by pid: egen age_mean = mean(age)
by pid: egen mstatus_mean = mean(mstatus)
\end{minted}

\begin{minted}{stata}
xtreg log_income edyears age i.male ib0.mstatus ib4.ethnicity ///
    i.child_birth age_mean mstatus_mean, re
\end{minted}

\tab{tab51.tex}{Correlated Random Effects (CRE) model}

\subquestion{a}{What is one advantage of the \textbf{CRE estimator} compared to the \textbf{random effects (RE)} estimator?}

\subquestion{b}{What is one advantage of the \textbf{CRE estimator} compared to the \textbf{fixed effects (FE)} estimator?}

\subquestion{c}{Compare the estimated coefficient for years of education from the \textbf{CRE model} with those from the \textbf{RE} and \textbf{FE} models. Are the coefficients similar or different? Explain why this is the case.}

\tab{comp5.tex}{POLS, RE, FE, CRE comparison}

\subquestion{d}{Based on your CRE estimates, does the assumption of \textbf{exogeneity} appear to hold? Which estimator would you consider most appropriate in this context?}